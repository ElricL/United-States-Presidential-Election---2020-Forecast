% Options for packages loaded elsewhere
\PassOptionsToPackage{unicode}{hyperref}
\PassOptionsToPackage{hyphens}{url}
%
\documentclass[
]{article}
\usepackage{lmodern}
\usepackage{amssymb,amsmath}
\usepackage{ifxetex,ifluatex}
\ifnum 0\ifxetex 1\fi\ifluatex 1\fi=0 % if pdftex
  \usepackage[T1]{fontenc}
  \usepackage[utf8]{inputenc}
  \usepackage{textcomp} % provide euro and other symbols
\else % if luatex or xetex
  \usepackage{unicode-math}
  \defaultfontfeatures{Scale=MatchLowercase}
  \defaultfontfeatures[\rmfamily]{Ligatures=TeX,Scale=1}
\fi
% Use upquote if available, for straight quotes in verbatim environments
\IfFileExists{upquote.sty}{\usepackage{upquote}}{}
\IfFileExists{microtype.sty}{% use microtype if available
  \usepackage[]{microtype}
  \UseMicrotypeSet[protrusion]{basicmath} % disable protrusion for tt fonts
}{}
\makeatletter
\@ifundefined{KOMAClassName}{% if non-KOMA class
  \IfFileExists{parskip.sty}{%
    \usepackage{parskip}
  }{% else
    \setlength{\parindent}{0pt}
    \setlength{\parskip}{6pt plus 2pt minus 1pt}}
}{% if KOMA class
  \KOMAoptions{parskip=half}}
\makeatother
\usepackage{xcolor}
\IfFileExists{xurl.sty}{\usepackage{xurl}}{} % add URL line breaks if available
\IfFileExists{bookmark.sty}{\usepackage{bookmark}}{\usepackage{hyperref}}
\hypersetup{
  pdftitle={United States Presidential Election - 2020 Forecast},
  pdfauthor={Elric Lazaro},
  hidelinks,
  pdfcreator={LaTeX via pandoc}}
\urlstyle{same} % disable monospaced font for URLs
\usepackage[margin=1in]{geometry}
\usepackage{longtable,booktabs}
% Correct order of tables after \paragraph or \subparagraph
\usepackage{etoolbox}
\makeatletter
\patchcmd\longtable{\par}{\if@noskipsec\mbox{}\fi\par}{}{}
\makeatother
% Allow footnotes in longtable head/foot
\IfFileExists{footnotehyper.sty}{\usepackage{footnotehyper}}{\usepackage{footnote}}
\makesavenoteenv{longtable}
\usepackage{graphicx,grffile}
\makeatletter
\def\maxwidth{\ifdim\Gin@nat@width>\linewidth\linewidth\else\Gin@nat@width\fi}
\def\maxheight{\ifdim\Gin@nat@height>\textheight\textheight\else\Gin@nat@height\fi}
\makeatother
% Scale images if necessary, so that they will not overflow the page
% margins by default, and it is still possible to overwrite the defaults
% using explicit options in \includegraphics[width, height, ...]{}
\setkeys{Gin}{width=\maxwidth,height=\maxheight,keepaspectratio}
% Set default figure placement to htbp
\makeatletter
\def\fps@figure{htbp}
\makeatother
\setlength{\emergencystretch}{3em} % prevent overfull lines
\providecommand{\tightlist}{%
  \setlength{\itemsep}{0pt}\setlength{\parskip}{0pt}}
\setcounter{secnumdepth}{-\maxdimen} % remove section numbering
\usepackage{amsmath}

\title{United States Presidential Election - 2020 Forecast}
\author{Elric Lazaro}
\date{November 2, 2020}

\begin{document}
\maketitle

\emph{Code and data supporting this analysis is available at:}\\
\url{https://github.com/ElricL/United-States-Presidential-Election-2020-Forecast.git}

\hypertarget{model}{%
\section{Model}\label{model}}

\hypertarget{model-specifics}{%
\subsection{Model Specifics}\label{model-specifics}}

~~~~~~Our model will be fitted using a survey sample obtained from
\href{https://www.voterstudygroup.org/publication/nationscape-data-set}{Democracy
Fund + UCLA Nationscape}. From the collection of data given, we will be
using the most recent survey available which is dated on June 25th,
2020. When processing the data I managed to create boolean (True/False)
variables of wether to vote Trump or not and as well as Biden. Such
variable translates well with logistic model regression. A logistic
model will be used as it allows us to estimate and isolate probability.
The following variables from our survey data will be used as I believe
there is a correlation between them and the potential candidate vote:
gender, race ethnicity, household income, and state. State in particular
will be used as a group variable, making our model a multilevel model.
This allows the model's intercept to vary from state to state. This is
done as I believe that different states have different starting points
in our model as there must be some correlation between the output of the
other predictor variables and the state. Therefore, using a group
variable should improve our model's performance in comparison to simply
generalizing our probability through all of United States. The logistic
regression model I am using can be written as a formula:

\begin{align*}
log(\frac{(p)}{1-(p)}) &= {\beta}_0 + 
                            {\beta}_1x_{gender : Male} + {\beta}_2x_{gender : Female} + \\
                        &    {\beta}_2x_{race\_ethnicity : Asian\: (Chinese)} + \\
                        &    {\beta}_3x_{race\_ethnicity : Asian\: (Japanese)} + \\
                        &    {\beta}_4x_{race\_ethnicity : Black\: or\: African\: American} + \\
                        &    {\beta}_5x_{race\_ethnicity : Other\: Asian\: or\: Pacific\: Islander} + \\
                        &    {\beta}_6x_{race\_ethnicity : Some\: other\: race} + \\
                        &    {\beta}_7x_{race\_ethnicity : White} + \\
                        &    {\beta}_8x_{household\_income : \$125000\: and\: above} + \\
                        &    {\beta}_9x_{household\_income : \$100000\: to\: \$124999} + \\
                        &    {\beta}_10x_{household\_income : \$75000\: to\: \$99999} + \\
                        &    {\beta}_11x_{household\_income : \$50000\: to\: \$74999} + \\
                        &    {\beta}_12x_{household\_income : \$15000\: to\: \$49999} + \\
                        &    {\beta}_13x_{household\_income : \$75000\: to\: \$99999} + \\
                        &    {\beta}_14x_{household\_income : Less\: than\: \$14999} + \epsilon\\
\end{align*}

~~~~~~Where \(p\) represents the probability that a voter or voters will
vote the model's specific candidate. Similarly, \(\beta_0\) represents
the intercept of the model, which will vary depending on the state.
Additionally, the following \(\beta\) represents the slopes of the
features. The higher the slope the higher the probability is for given
variable. For instance if the slope for a person with household income
`\(Less\: than\: \$14,999\)' is -1 while one with income of
\(\$15,000\: to\: \$49,999\) is 1, the person with higher income is more
likely to vote the model's candidate given their positive and large
slope. Note that a negative slope also decreases likelihood while a
positive increases.\\
\hspace*{0.333em}\hspace*{0.333em}\hspace*{0.333em}\hspace*{0.333em}\hspace*{0.333em}\hspace*{0.333em}Note
that since we are analyzing votes for Trump and Biden we will perform
this model twice with a fit for each candidate.

\hypertarget{post-stratification}{%
\subsection{Post-Stratification}\label{post-stratification}}

~~~~~~In order to estimate the proportion of voters who will vote for
either candidates I need to perform a post-stratification analysis. The
estimation for our logistic model will be done on
\href{https://usa.ipums.org/usa/index.shtml}{American Community Surveys
(ACS) data}. This census data will provide us the Individual/Unit level
features that can be entered into our model along with the individual's
State, allowing us to make great amount of predictions. Predictions will
come in form of cells based of diefferent gender, race ethnicity,
household income, and State. Each will then be weighted based on their
population size divided by the entire population size. Note that for
post-stratification we can only use one model. We can simply choose the
model that has a better correlation between our probability of vote and
the variables. This can be achieved by observing the model's p-values
and which one is usually lower. We can do this as a lower p-value means
we can more likely reject the null hypothesis that there is no
correlation.

\hypertarget{results}{%
\section{Results}\label{results}}

The following are the results of our model fit for each candidate:

\begin{longtable}[]{@{}lrr@{}}
\toprule
factors & Trump\_Estimate & Biden\_Estimate\tabularnewline
\midrule
\endhead
gender: Male & 0.40 & -0.37\tabularnewline
race\_ethnicity: Asian (Chinese) & -1.33 & 1.33\tabularnewline
race\_ethnicity: Asian (Japanese) & -1.14 & 1.50\tabularnewline
race\_ethnicity: Black or African American & -2.06 & 1.83\tabularnewline
race\_ethnicity: Other Asian or Pacific Islander & -0.71 &
0.79\tabularnewline
race\_ethnicity: Some other race & -0.76 & 0.76\tabularnewline
race\_ethnicity: White & 0.06 & 0.21\tabularnewline
household\_income: \$125,000 and above & 0.12 & 0.00\tabularnewline
household\_income: \$15,000 to \$49,999 & -0.27 & 0.28\tabularnewline
household\_income: \$50,000 to \$74,999 & -0.21 & 0.25\tabularnewline
household\_income: \$75,000 to \$99,999 & -0.25 & 0.39\tabularnewline
household\_income: Less than \$14,999 & -0.47 & 0.21\tabularnewline
\bottomrule
\end{longtable}

\textbf{Figure 1:} Summary of the logistic model for the likelihood of
voting for Donald Trump and Jose Biden. Shows key details as to how each
characteristic of an Individual affects their probability of voting the
candidate.

~~~~~~The \(\beta\) coefficients or slopes of our models can be
identified in the estimate column which we can then use to observe how
each individual characteristic affect the probability of voting the
candidate. The fit of our model has concluded that individuals with
certain characteristics will more likely vote one than the other. For
instance, we can observe that the estimate/{[}Beta{]} coefficient for
male gender has a positive number for Trump's logistic model while
Biden's is negative. We should be able to observe from this result that
there is a higher chance that male-identified people will vote Trump
over Biden. More on the results of these estimates are further detailed
in the Discussion section.\\
\hspace*{0.333em}\hspace*{0.333em}\hspace*{0.333em}\hspace*{0.333em}\hspace*{0.333em}\hspace*{0.333em}There
were more lower p-values in Biden's model than Trump's and as a result I
am more confident with rejecting the null hypothesis of no correlation
between the logistic probability and the variables for Biden's logistic
model. Therefore, Post-stratification was estimated using logistic model
on probability of voting Biden.

\begin{longtable}[]{@{}lrr@{}}
\toprule
Parameter & Trump\_p\_value & Biden\_p\_value\tabularnewline
\midrule
\endhead
b\_Intercept & 0.6840 & 0.0455\tabularnewline
b\_as.factorgenderMale & 0.0000 & 0.0000\tabularnewline
b\_as.factorrace\_ethnicityAsianChinese & 0.0040 & 0.0010\tabularnewline
b\_as.factorrace\_ethnicityAsianJapanese & 0.0755 &
0.0135\tabularnewline
b\_as.factorrace\_ethnicityBlackorAfricanAmerican & 0.0000 &
0.0000\tabularnewline
b\_as.factorrace\_ethnicityOtherAsianorPacificIslander & 0.0340 &
0.0200\tabularnewline
b\_as.factorrace\_ethnicitySomeotherrace & 0.0170 &
0.0140\tabularnewline
b\_as.factorrace\_ethnicityWhite & 0.8080 & 0.4710\tabularnewline
b\_as.factorhousehold\_income.125000andabove & 0.3455 &
0.9880\tabularnewline
b\_as.factorhousehold\_income.15000to.49999 & 0.0150 &
0.0160\tabularnewline
b\_as.factorhousehold\_income.50000to.74999 & 0.0830 &
0.0560\tabularnewline
b\_as.factorhousehold\_income.75000to.99999 & 0.0515 &
0.0060\tabularnewline
b\_as.factorhousehold\_incomeLessthan.14999 & 0.0005 &
0.1295\tabularnewline
\bottomrule
\end{longtable}

\textbf{Figure 3:} p-values for each coefficient in Trump's and Biden's
logistic Model.

~~~~~~I estimated that the proportion of voters in favor of voting for
Joe Biden to be 0.6059. This is based off our post-stratification
analysis of the proportion of voters in favor of Joe Biden modeled by a
multilevel logistic regression model, which accounted for gender, racial
ethnicity, and household income. We can also then assume that Trump will
likely have lesser proportion of 0.3941\\
\hspace*{0.333em}\hspace*{0.333em}\hspace*{0.333em}\hspace*{0.333em}\hspace*{0.333em}\hspace*{0.333em}When
grouping the post-stratification values by states, we can also see that
the prediction of majority of the votes going to Joe Biden is reflected
for all states.

\begin{longtable}[]{@{}lrr@{}}
\toprule
state & biden\_mean & trump\_mean\tabularnewline
\midrule
\endhead
AK & 0.5915968 & 0.4084032\tabularnewline
AL & 0.6083066 & 0.3916934\tabularnewline
AR & 0.5957102 & 0.4042898\tabularnewline
AZ & 0.5943875 & 0.4056125\tabularnewline
CA & 0.6222799 & 0.3777201\tabularnewline
CO & 0.6058778 & 0.3941222\tabularnewline
CT & 0.6201915 & 0.3798085\tabularnewline
DE & 0.6192934 & 0.3807066\tabularnewline
FL & 0.6043012 & 0.3956988\tabularnewline
GA & 0.6069732 & 0.3930268\tabularnewline
HI & 0.6295420 & 0.3704580\tabularnewline
IA & 0.5988385 & 0.4011615\tabularnewline
ID & 0.5853934 & 0.4146066\tabularnewline
IL & 0.6171970 & 0.3828030\tabularnewline
IN & 0.6001897 & 0.3998103\tabularnewline
KS & 0.5894536 & 0.4105464\tabularnewline
KY & 0.6062190 & 0.3937810\tabularnewline
LA & 0.6202668 & 0.3797332\tabularnewline
MA & 0.6232733 & 0.3767267\tabularnewline
MD & 0.6288788 & 0.3711212\tabularnewline
ME & 0.6036194 & 0.3963806\tabularnewline
MI & 0.6148379 & 0.3851621\tabularnewline
MN & 0.6076186 & 0.3923814\tabularnewline
MO & 0.6021359 & 0.3978641\tabularnewline
MS & 0.6209450 & 0.3790550\tabularnewline
MT & 0.5954066 & 0.4045934\tabularnewline
NC & 0.6123569 & 0.3876431\tabularnewline
ND & 0.5931499 & 0.4068501\tabularnewline
NE & 0.6048072 & 0.3951928\tabularnewline
NH & 0.5993558 & 0.4006442\tabularnewline
NJ & 0.6100697 & 0.3899303\tabularnewline
NM & 0.6061043 & 0.3938957\tabularnewline
NV & 0.6011792 & 0.3988208\tabularnewline
NY & 0.6204264 & 0.3795736\tabularnewline
OH & 0.6057539 & 0.3942461\tabularnewline
OK & 0.5964378 & 0.4035622\tabularnewline
OR & 0.6087793 & 0.3912207\tabularnewline
PA & 0.5965854 & 0.4034146\tabularnewline
RI & 0.6083086 & 0.3916914\tabularnewline
SC & 0.5955919 & 0.4044081\tabularnewline
SD & 0.5939414 & 0.4060586\tabularnewline
TN & 0.5923052 & 0.4076948\tabularnewline
TX & 0.5913462 & 0.4086538\tabularnewline
UT & 0.5881083 & 0.4118917\tabularnewline
VA & 0.6190287 & 0.3809713\tabularnewline
VT & 0.6099102 & 0.3900898\tabularnewline
WA & 0.6187212 & 0.3812788\tabularnewline
WI & 0.6105218 & 0.3894782\tabularnewline
WV & 0.5947574 & 0.4052426\tabularnewline
WY & 0.5975353 & 0.4024647\tabularnewline
\bottomrule
\end{longtable}

\textbf{Figure 3:} Post-stratification values of proportion voters for
Joe Biden and Donald Trump per State.

\hypertarget{discussion}{%
\section{Discussion}\label{discussion}}

~~~~~~Using my multilevel logistic regression model that I fitted for
both Trump vote and Biden vote intentions, I was able to observe which
kind of individuals would gravitate towards Joe Biden or Trump based on
their gender, ethnicity, and household income. With the two regression
models, I chose Biden's model for our post-stratification process for
having better correlation with my individual variables. Using the census
data obtained from American Community Surveys, we then estimated on
different cells based on the identified gender, race ethnicity, and
household income.\\
\hspace*{0.333em}\hspace*{0.333em}\hspace*{0.333em}\hspace*{0.333em}\hspace*{0.333em}\hspace*{0.333em}Using
the slopes that we have observed from our results we can make deductions
on how each variable affect the likelihood of an individual voting for
either candidate. Starting with trump I can observe the following:

\begin{itemize}
\item
  Race ethnicity other than `White' have lesser likelihood of voting
  him.
\item
  Males are more likely to vote Trump compared to Females.
\item
  Individuals with high income (\$125,000 and above) are more likely to
  vote him while a specified lesser income decreases the chances.\\
\end{itemize}

Biden on the other hand had the opposite situation with the following
relationships:

\begin{itemize}
\item
  Non `White' ethnicities have likelier chance of voting Biden. However,
  do note that individuals identified as `White' still have decent
  chance of voting Biden given their positive slope.
\item
  Males are less likely to vote Biden compared to females.
\item
  Individuals with income less than \$125,000 are more likely to vote
  Biden.
\end{itemize}

~~~~~~However, it appeared that overall there were more characteristics
that leads to a higher chance of favoring Biden. This can be further
strengthened by my post-stratification results which gave a higher
estimate for Joe Biden. Based off the estimated proportion of voters in
favor of voting for Joe Biden being 0.6059, I predict that Joe Biden
will win the election.

\hypertarget{weaknesses}{%
\subsection{Weaknesses}\label{weaknesses}}

~~~~~~There are some limitations to my research. Given how expensive
fitting a logistic model with Bayesian inference, variables such as
household income had to be simplified. For same reason, age had to be
omitted as a census dataset with it proved to be too large to calculate
predictions on our model. The census data and survey data had some
differences with their outputs resulting in needing to omit some
information to generalize and match the two datasets. For instance,
survey data contained more types of racial ethnicities but since we need
to make predictions based off census data which had more limited output,
many of the races had to be re-categorized. i.e.~Asian (Asian Indian)
outputs in the survey data had to be replaced with `Other Asian or
Pacific Islander'. Lastly, while the model with best p-values were
chosen for post-stratification process, some were still above the
standard alpha level of 0.05 and as a result it is possible correlation
may be weak.

\hypertarget{next-steps}{%
\subsection{Next Steps}\label{next-steps}}

~~~~~~To further improve the model and post-stratification process, we
can compare it to the final election results. If our prediction on Joe
Biden winning the election ends up being wrong, it is critical to see
where our analysis went wrong. There are many things we can look into
that would result in our poor prediction in such scenario. Firstly, We
could have missed an important variable in our model. Our dataset may
not reflect well with general population. Further samples and surveys
should have been used. Other ways we could have improve this model if
given wrong prediction is by trying out different models if logistic
regression ends up being inaccurate fit. For instance, using a model
with Bayesian Inference may have allowed too much bias in the fit.
Testing more efficient models may also be useful as it would allow me to
use age as one of predictor variables as originally intended. Survey
results also had ``Don't know'' or ``Someone else'' as the individual's
vote intention. While their occurrence frequency should be too few to
affect our final prediction result, it may still be interesting to find
how these individuals have an affect on the outcome.

\hypertarget{references}{%
\section{References}\label{references}}

\begin{enumerate}
\def\labelenumi{\arabic{enumi}.}
\item
  Join two tbls together. (n.d.). Retrieved from
  \url{https://dplyr.tidyverse.org/reference/join.html}
\item
  Extract or Replace Parts of a Data Frame. (n.d.). Retrieved November
  02, 2020, from
  \url{https://astrostatistics.psu.edu/su07/R/html/base/html/Extract.data.frame.html}
\item
  Rename Data Frame Columns in R. (n.d.). Retrieved November 02, 2020,
  from
  \url{https://www.datanovia.com/en/lessons/rename-data-frame-columns-in-r/}
\item
  Convert case of a string - case. (n.d.). Retrieved November 02, 2020,
  from \url{https://stringr.tidyverse.org/reference/case.html}
\item
  Hadley Wickham {[}aut, C. (2020, August 18). Group\_by\_all: Group by
  a selection of variables in dplyr: A Grammar of Data Manipulation.
  Retrieved November 02, 2020, from
  \url{https://rdrr.io/cran/dplyr/man/group_by_all.html}
\item
  Daniel Lüdecke {[}aut, C. (2020, October 29). P\_value: P-values in
  parameters: Processing of Model Parameters. Retrieved November 02,
  2020, from \url{https://rdrr.io/cran/parameters/man/p_value.html}
\item
  Press, C., Finance, Y., \& Newsweek. (2020, October 30). New: Second
  Nationscape Data Set Release. Retrieved November 02, 2020, from
  \url{https://www.voterstudygroup.org/publication/nationscape-data-set}
\item
  Team, M. (n.d.). U.S. CENSUS DATA FOR SOCIAL, ECONOMIC, AND HEALTH
  RESEARCH. Retrieved November 02, 2020, from
  \url{https://usa.ipums.org/usa/index.shtml}
\end{enumerate}

\end{document}
